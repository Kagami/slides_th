\documentclass{beamer}
\usepackage[utf8]{inputenc}
\usepackage[russian]{babel}
\usepackage{minted}

\setbeamertemplate{navigation symbols}{}

\begin{document}

\newminted{haskell}{fontsize=\footnotesize}

\title{Основы Template Haskell}
\date{Октябрь 2012 г.}
\maketitle

\begin{frame}{Описание}
\begin{itemize}
    \item Compile-time meta-programming
    \item Разработан SPJ и Tim Sheard в 2002-ом году (MSR)
    \item Реализован в GHC с версии 6 (и только в нём)
    \item Ближайшие аналоги
    \begin{itemize}
        \item LISP (runtime)
        \item Python compiler/ast модули (runtime)
    \end{itemize}
    \item Похожие инструменты
    \begin{itemize}
        \item Препроцессоры (cpp, perl source filters, etc.)
        \item eval
    \end{itemize}
\end{itemize}
\end{frame}

\begin{frame}{Преимущества}
\begin{itemize}
    \item Позволяет делать вещи, которые
    не реализовать другими средствами (DSL)
    \item Помогает избавиться от boilerplate-кода (auto derives)
\end{itemize}
\end{frame}

\begin{frame}[fragile]{Синтаксис}
\begin{itemize}
    \item Прагма -XTemplateHaskell
    \item Language.Haskell.TH, Language.Haskell.TH.Syntax
    \item Q Monad
    \item Name
    \begin{haskellcode}
    Prelude Language.Haskell.TH> :t 'True
    'True :: Name
    Prelude Language.Haskell.TH> 'True
    GHC.Types.True
    Prelude Language.Haskell.TH> ''Maybe
    Data.Maybe.Maybe
    \end{haskellcode}
    \item \$splice
    \begin{haskellcode}
    $(deriveShow ''Test) :: Q [Dec]
    $(tuple 10 "test") :: Q Exp
    $(generateType 'Int 'Bool) :: Q Type
    \end{haskellcode}
\end{itemize}
\end{frame}

\begin{frame}[fragile]{Quotation}
\begin{itemize}
    \item {[}| … |{]}, {[}e| … |{]} :: Q Exp
    \begin{haskellcode}
    [e| 1 + 2 + 3 |]

    Prelude Language.Haskell.TH> runQ [| 1 + 2 + 3|]
    InfixE (Just (InfixE (Just (LitE (IntegerL 1)))
    (VarE GHC.Num.+) (Just (LitE (IntegerL 2)))))
    (VarE GHC.Num.+) (Just (LitE (IntegerL 3)))
    \end{haskellcode}
    \item {[}d| … |{]} :: Q {[}Dec{]}
    \begin{haskellcode}
    [d| const a b = a |]
    \end{haskellcode}
    \item {[}t| … |{]} :: Q Type
    \begin{haskellcode}
    [t| Int -> String -> Bool |]
    \end{haskellcode}
    \item {[}p| … |{]} :: Q Pat
    \begin{haskellcode}
    [p| (_, []) |]
    \end{haskellcode}
\end{itemize}
\end{frame}

\begin{frame}[fragile]{Quasi-quotation }
\begin{itemize}
    \item GHC 6.10+
    \item Прагма -XQuasiQuotes
    \item
    \begin{haskellcode}
    [quoter| string |]
    \end{haskellcode}
    \item
    \begin{haskellcode}
    data QuasiQuoter = QuasiQuoter
        { quoteExp  :: String -> Q Exp
        , quotePat  :: String -> Q Pat
        , quoteType :: String -> Q Type
        , quoteDec  :: String -> Q [Dec]
        }
    \end{haskellcode}
    \item
    \begin{haskellcode}
    [persist|
        Person
            name String
            age Int
            deriving Show
        Car
            color String
            make String
            model String
            deriving Show
    |]
    \end{haskellcode}
\end{itemize}
\end{frame}

\begin{frame}[fragile]{Пример 1}
\inputminted[fontsize=\tiny]{haskell}{examples/Main1.hs}
\begin{haskellcode*}{fontsize=\tiny}
---------------------------------------
% runghc -ddump-splices Main1.hs

InfixE (Just (LitE (IntegerL 7))) (VarE GHC.Num.*) (Just (LitE (IntegerL 3)))

Main.hs:5:7-25: Splicing expression
    compile [| 7 * 3 |] ======> 21

InfixE (Just (InfixE (Just (LitE (IntegerL 7))) (VarE GHC.Num.*) (Just (LitE (IntegerL 9)))))
(VarE GHC.Num.+) (Just (LitE (IntegerL 3)))

Main.hs:6:7-29: Splicing expression
    compile [| 7 * 9 + 3 |] ======> 66

InfixE (Just (InfixE (Just (LitE (IntegerL 5))) (VarE GHC.Num.*) (Just (LitE (IntegerL 8)))))
(VarE GHC.Num.+)
(Just (InfixE (Just (LitE (IntegerL 6))) (VarE GHC.Num.*) (Just (LitE (IntegerL 3)))))

Main.hs:7:7-33: Splicing expression
    compile [| 5 * 8 + 6 * 3 |] ======> 58
\end{haskellcode*}
\end{frame}

\begin{frame}[fragile]{Пример 1 (код)}
\inputminted[fontsize=\tiny]{haskell}{examples/TH1.hs}
\end{frame}

\begin{frame}[fragile]{Пример 2}
\inputminted[fontsize=\tiny]{haskell}{examples/Main2.hs}
\begin{haskellcode*}{fontsize=\tiny}
---------------------------------------
% runghc -ddump-splices Main2.hs
Main2.hs:1:1: Splicing declarations
    deriveShow ''Test
  ======>
    Main2.hs:10:3-19
    instance Show Test where
      show rec_a1jH
        = ("Main.Test:\
           \\t"
           ++
             (Data.List.intercalate
                "\
                \\t"
                [("Main.fieldA: " ++ (show (fieldA rec))),
                 ("Main.fieldB: " ++ (show (fieldB rec)))]))
Main.Test:
    Main.fieldA: 1
    Main.fieldB: False
\end{haskellcode*}
\end{frame}

\begin{frame}[fragile]{Пример 2 (код)}
\inputminted[fontsize=\tiny]{haskell}{examples/TH2.hs}
\end{frame}

\begin{frame}{Примеры проектов}
\begin{itemize}
    \item Yesod (models, templates, routes, etc.)
    \item Happstack (models)
    \item TwilightSparkle (models, auto derives)
\end{itemize}
\end{frame}

\begin{frame}{Недостатки}
\begin{itemize}
    \item Не type-safe, не haskell-way
    \begin{itemize}
        \item Что находится внутри Q {[Dec]}, Q Exp ?
        \item Можно генерировать нерабочий код и выяснить это только
        при непосредственном использовании
        \item Очень сложно писать unit-тесты
        (в том числе из-за ограничений reify)
        \item Имеет доступ к приватным определениям модулей
        \item Лишь незначительно лучше препроцессора
        (не нужно разбирать AST)
        \item QQ работает с обычной строкой, для которой надо писать
        свой парсер, что делает его поведении совершенно некотролируемым
    \end{itemize}
    \item Сложно читать код
    \item Страшно выглядит
    \item Может выполнять при {\it компиляции} произвольный код
    \item TH-функции нельзя использовать в том же модуле,
    где они определены
\end{itemize}
\end{frame}

\begin{frame}{Ссылки}
\begin{itemize}
    \item \url{http://www.haskell.org/haskellwiki/Template_Haskell}
    \item \url{http://www.haskell.org/ghc/docs/latest/html/users_guide/template-haskell.html}
    \item \url{http://www.haskell.org/haskellwiki/Quasiquotation}
    \item \url{http://hackage.haskell.org/packages/archive/template-haskell/latest/doc/html/Language-Haskell-TH.html}
    \item \url{http://www.yesodweb.com/book/persistent}
    \item \url{http://www.yesodweb.com/blog/2012/10/yesod-pure}
    \item \url{http://stackoverflow.com/questions/10857030/whats-so-bad-about-template-haskell}
\end{itemize}
\end{frame}

\begin{frame}
  \begin{center}
    \Large Вопросы?
  \end{center}
\end{frame}

\end{document}
